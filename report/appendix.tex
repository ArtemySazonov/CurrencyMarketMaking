\begin{appendices} % Do not change this line (if you have appendices). 
    \section{Data collection and  order book implementation}\label{appendix:data}
        The estimated size of the data is approximately 80 Gb. The data is stored in the JSON format and has the following fields:
        \begin{itemize}
            \item \mintinline{rust}|date| contains a string with an information about the tick itself.
            \item \mintinline{rust}|instrument| contains a name of the currency pair.
            \item \mintinline{rust}|r#type| contains a enum: either a \mintinline{rust}|SNAPSHOT| or an \mintinline{rust}|INCREMENT|.
            \item \mintinline{rust}|side| contains a enum: either \mintinline{rust}|BID| or \mintinline{rust}|ASK|.
            \item \mintinline{rust}|quotes|:
                \begin{itemize}
                    \item if \mintinline{rust}|type| is \mintinline{rust}|SNAPSHOT|, then this field contains an array of L3 quotes, i.e. status quo of the side of the book;
                    \item if \mintinline{rust}|type| is \mintinline{rust}|INCREMENT|, then this field contains two arrays of L3 quotes and an array of \mintinline{rust}|int|s (\mintinline{rust}|added|, \mintinline{rust}|changed|, and \mintinline{rust}|deleted| correspodingly).
                \end{itemize}
        \end{itemize}
        An example of a tick: \begin{minted}[autogobble]{json}
            {
                "date":       "2022-11-10T07:00:03.123000001",
                "instrument": "USD/RUB_T+1",
                "type":       "INCREMENT",
                "side":       "ASK",
                "quotes":
                {
                    "added": [
                        {
                            "id":    615102,
                            "price": 61.5125,
                            "size":  50000.0
                        }
                    ],
                    "changed": [
                        {
                            "id":    615101,
                            "price": 61.2150,
                            "size":  10000.0
                        }
                    ],
                    "deleted": [ 615202 ]
                }
            }
        \end{minted}

        We implemented the order book in Rust using the B-tree as a base data structure (See \Cite{Cormen2022}). To our knowledge, 
        the ring buffer is considered to be the best practice in the industrial tasks for the order book storing, but we used 
        the B-tree due to the ease of the implementation.
        In order to effectively search the order book, we implemented the hash map contating the IDs of the quotes and
        the price levels at which the order is placed. The interface of the class looks like this:
        \begin{minted}[autogobble]{rust}
            pub struct L3Quote
            {
                pub id: i64,
                pub price: f64,
                pub size: f64,
            }

            pub struct OrderBook
            {
                pub instrument: String,
                pub bid: BTreeMap<i64, VecDeque<L3Quote>>,
                pub ask: BTreeMap<i64, VecDeque<L3Quote>>,

                price_step: f64,
                price_step_inv: f64,
            }

            impl OrderBook
            {
                pub fn new(price_step: f64) -> OrderBook { ... }
                pub fn update(&mut self, lines: Vec<String>) { ... }
                pub fn from_file(filename: &str) -> Result<OrderBook, String> { ... }
                pub fn print(orderbook: &OrderBook) { ... }
            }

            impl Iterator for OrderBook 
            {
                type Item = Order; 
                fn next(&mut self) -> Option<Self::Item> { ... }
            }
        \end{minted}
        
        The \mintinline{rust}|OrderBook| struct contains the following fields:
        \begin{enumerate}
            \item \mintinline{rust}|instrument|: a field containing the name of the asset;
            \item \mintinline{rust}|bid| and \mintinline{rust}|ask|: sides of an order book. They are 
                stored as a B-tree with price levels as keys and queues of L3 quotes as the items correspoding
                to the price levels.
            \item \mintinline{rust}|price_step| and \mintinline{rust}|price_step_inv| are needed to calculate the 
                price level of the correspoding order. \mintinline{rust}|key = price*price_step_inv|.
        \end{enumerate}
        The data is read from the raw MOEX files using the \href{github.com/serde-rs/json}{\texttt{serde-rs/json}} library. Then, the obtained data is
        converted into the \mintinline{rust}|OrderBook| structure tick-by-tick using the \mintinline{rust}|OrderBook::update| method.

    \section{Additional backtesting}\label{appendix:backtest}
        Here we used the first set of parameters for the backtest from Section \ref{section:backtest}.
        \subimport{./}{backtest.tex}
\end{appendices}   % Do not change this line