\introduction % Do not change this line
    Trade execution plays a crucial role in financial markets, impacting both profitability and market efficiency. In 
    today's fast-paced electronic markets, achieving optimal trade execution has become increasingly challenging. With 
    the rapid growth of online trading platforms and the emergence of high-frequency trading, market participants must 
    continually adapt to dynamic market conditions to maximize their returns.

    When it comes to electronic currency markets, the need for optimal trade execution becomes even more critical. The 
    continuous nature of currency trading, combined with the high liquidity and volatility, necessitates efficient 
    execution to mitigate transaction costs and minimize market impact. As a result, market participants are constantly 
    seeking innovative approaches to enhance their execution strategies and gain a competitive edge.

    One promising avenue for achieving optimal trade execution in electronic markets is through some machine learning 
    techniques. Machine learning algorithms have shown great potential in analyzing complex data patterns, identifying 
    market dynamics, and adapting to changing conditions in real-time. Online machine learning approach offers the 
    advantage of updating the models on the fly, incorporating new information as it becomes available, and continuously 
    improving the performance metrics of the execution strategies.

    However, developing an optimal execution algorithm comes with its own set of challenges. Factors such as market 
    microstructure, liquidity constraints, time-sensitive decision-making, and risk management need to be carefully 
    considered in the algorithm's design. Furthermore, the trade-off between minimizing the execution time (i.e. minimizing the market risk we would like to hold)
    and minimizing the price impact poses a significant dilemma, requiring a nuanced understanding of the market behavior and execution goals.

    In this paper, we focus on the implementation of an online machine learning approach to achieve optimal trade execution 
    in electronic markets. We explore the potential benefits and challenges of such an approach and provide insights into the 
    factors that contribute to successful execution in electronic markets.

    The remainder of this paper is organized as follows: 
    Section \ref{section:OMLandRL} provides an overview of the foundations of online learning and reinforcement learning for the optimal trade execution needs. 
    Section \ref{section:ElectronicMM} introduces basic market microstructure terminology and states the optimal execution problem.
    Section \ref{section:OptExecAlgos} describes the methodology and framework of our execution approach. 
    Section \ref{section:Backtest} presents the results of our empirical analysis based on real market data and discusses the performance and limitations of implemented algorithms. 
    Finally, the last section concludes the paper, summarizing the key findings and outlining potential directions for future research in this area.