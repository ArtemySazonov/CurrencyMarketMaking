\documentclass[aspectratio=169]{beamer}
\usetheme{vega}

\usepackage{graphicx}
\DeclareGraphicsExtensions{.pdf,.png,.jpg}
% \beamertheme[russian]
% \usetheme{vega}

% \addbibresource{assets/pricing_lib.bib}

\DeclareMathOperator*{\plim}{\ensuremath{\operatorname{\P-lim}}}

\newcommand{\cA}{\mathcal{A}}
\newcommand{\cB}{\mathcal{B}}
\newcommand{\cC}{\mathcal{C}}
\newcommand{\cN}{\mathcal{N}}

\subtitle{Студенческая научная группа <<Валютный маркет мейкинг>>}
\title{To sell or not to sell? That is the question.}
\author{Артемий Сазонов, Данил Лёгенький}
\institute{МГУ имени М.В. Ломоносова, механико-математический факультет}
\date{Июль 24, 2023}

\begin{document}


    \maketitle
    \section{ Немного о задаче}
    
    % \begin{frame}{Терминология}
    %     Большинство ликвидных рынков являются электронными и используют 
    %     используют механизм двойного аукциона с непрерывным временем, использующий книгу лимитных ордеров (LOB), в которой транзакция происходит всякий раз, когда покупатель и продавец 
    %     "договариваются" о цене.

    % \end{frame}

    \begin{frame}{Терминология}
        Ордер — это заявленное в определенное время подачи обязательство купить или продать определенный объем актива по цене не ниже заданной.

        Ордера, которые совпадают по прибытии, называются рыночными ордерами. Ордера, которые не совпадают по прибытии, называются лимитными ордерами. Математически, ордер может быть описан вектором, состоящим из 4-х параметров: $x = (\varepsilon_x, p_x, v_x, t_x) $, где $\varepsilon_x = \pm 1 $ ($\varepsilon_x = + 1$ - означает, что данный ордер - ордер на покупку, и, соответственно, $\varepsilon_x = - 1$ - ордер на продажу), $p_x$ - цена, $v_x$ - объем, $t_x$ - время подачи ордера.

    \end{frame}
    
    \begin{frame}{Терминология}
        Книга лимитных ордеров (LOB) — это просто 
        набор выявленных неудовлетворенных намерений купить или продать актив в определенный момент времени.

        Точнее, LOB $L(t)$ — это набор всех лимитных ордеров для данного актива на данной платформе в момент времени $t$.   

        LOB можно представить в следующем виде
        \begin{equation*}
            L(t) = \mathcal{A}(t) \sqcup \mathcal{B}(t),
        \end{equation*}
        где $\mathcal{A}(t)$ - набор всех лимитных ордеров на продажу, $\mathcal{B}(t)$ - набор всех лимитных ордеров на покупку. 
    \end{frame}

    \begin{frame}{Терминология}
    
        \begin{itemize}
        
            \item Цена спроса (bid) в момент времени $t$ является самой высокой заявленной ценой среди лимитных ордеров на покупку в момент времени $t$: 
            \begin{equation*}
                p_b(t) = \max\limits_{x \in \mathcal B(t)} p_x.
            \end{equation*}
    
            \item Цена предложения (ask) в момент времени $t$ является самой низкой заявленной ценой среди лимитных ордеров на продажу в момент времени $t$:
            \begin{equation*}
                p_a(t) = \min\limits_{x \in \mathcal A(t)} p_x.
            \end{equation*}
    
            \item  Средняя цена (mid) в момент времени $t$ равна
            \begin{equation*}
                p_m(t) = \frac{p_a(t) + p_b(t)}{2}.
            \end{equation*}
            
        \end{itemize} 
    \end{frame}

    \begin{frame}{Терминология}

        \begin{itemize}
        
            \item Спред спроса и предложения (bid-ask spread) (или просто «спред» (spread) ) в момент времени $t$ определяется:
            \begin{equation*}
                s(t) = p_a(t) - p_b(t).
            \end{equation*}
            
        \end{itemize}

    \end{frame}


    % \begin{frame}{Постановка задачи}
    %     1) Описать кто такие маркет-мейкеры в современном рынке.
        
    %     2) В чем риск для маркет-мейкеров (информированая/неиформированная тороговля)
        
    %     3) price-impact
        
    %     4) optimal execution
    % \end{frame}

    \begin{frame}{Постановка задачи}
        Рассмотрим следующую ситуацию: мы хотим продать/купить какое-то количество некоторого актива, посылая рыночный ордер.
        
        Если объем нашего рыночный ордер больше суммарного объема лимитных ордеров по наилучшей цене, то оставшаяся часть рыночного ордера будет испольняться с помощью вторых по величине лимитных ордеров из LOB. Более того, мы даем остальным участникам информацию он наших намерениях. Из-за всего этого текущая рыночная цена актива движется в направлении рыночного ордера. Таким образом действия трейдера могут влиять на рыночную цену актива.
        Данный эффект называется влиянием на цену (price impact).

    \end{frame}

    \begin{frame}{Постановка задачи}
        Кажется, что данной проблемы можно избежать, разбив наш большой рыночный ордер на множество небольших ордеров, и посылая наши ордера таким образом, чтобы мы всегда находились в пределах лучшего бида/аска. Однако, в этом случае мы берем на себя риск того, что за время пока мы будем продавать/покупать, цена актива может либо сильно вырасти либо сильно упасть. Такой риск называется рыночным. 

        Так и поялвяется задача оптимального исполнения.

        Более формально, задача оптимального исполнения формулируется следующим образом: ликвидация позиции, минимизируя некоторый функционал от влияния на цену и рыночного риска. 
    \end{frame}


    \section{Простые алгоритмы}

    \begin{frame}{Общее определение}
        Предположим, что у нас есть $W$ лотов некоторого актива, который мы хотим продать за время $T$. Разделим $T$ на $L$ интервалов длины $\tau = T/L$, и определим $t_k = k\tau, \; k = 0,1, \ldots, L$.
    
        \begin{block}{Определение}
            Торговая траектория - это процесс $(w_k)_{k = 0, \ldots, L}$, где $w_k$ - число лотов, которые мы продолжаем удерживать момент времени $t_k$. 
            
            Определить тороговую траекторию, это тоже самое, что определить торговый лист $(A_k)_{k \in \mathcal L}, \; \mathcal L = \{1,2,\ldots, L \}$, где $A_k$ - число лотов, которые мы продаем в момент времени $t_k$. 
    
            Торговая стратегия - это правило, по которому определяется $A_k$ в момент времени $t_k$ по имеющийся информации к этому моменту времени. 
        \end{block}
    
    \end{frame}

    
    \begin{frame}{Time-Weighted Average Price}

        Один из самых простых алгоритмов, которые можно придумать, это, конечно же, алгоритм, при котором $A_i = A_j, \; \forall i, j \in \mathcal L$. 
    
        Вполне очевидно, что при отсутствии рыночного риска, данная стратегия была бы оптимальной.
        

    \end{frame}

    \begin{frame}{Almgren-Chriss}
        Теперь пусть $\zeta_l, \; l \in \mathcal L$ - н.о.р. с 0 средним и единичной дисперсией. Пусть также $g(A_l), \; h(A_l)$ - функции постоянного и временного воздействия на цену соответственно. Тогда предполагается, что динамика бида задается как случайное блуждание:
        
        \begin{align*}
            &p_b(l) = p_b(l - 1) + \sigma \zeta_l - g(A_l), \\
            &\tilde p_b(l) = p_b(l - 1) - h(A_l),
        \end{align*}
        здесь $\tilde p_b(l)$ - эффективная цена, $\sigma > 0$.

    \end{frame}

    \begin{frame}{Almgren-Chriss}

         Немного поясним, что имеется ввиду под временным и постоянным воздейтвием на цену
    
         Временное воздействие на цену приводит к временному отклонению цены от ее равновесия из-за нашей торговой стратегии, которое исчезает в течение следующего торгового периода. 
         
         Постоянное влияние на цену относится к изменению равновесной цены из-за нашей торговой стратегии, которое длится, по крайней мере, до конца раунда. 
    
    \end{frame}

    \begin{frame}{Almgren-Chriss}

        \begin{block}{Определение}
            Определим фиксацию траектории как общий номинальный торговый доход по завершении исполнения  
            \begin{equation*}
                CP( A_{\cdot}, S_{\cdot}) = \sum_{l=1}^L A_{l}S_l,
            \end{equation*}
            здесь $S_{\cdot}$ - процесс цены актива. 

            Общие издержки торговой траектории определяется следующим образом:

            \begin{equation*}
                \mathcal P = \mathcal P(W, A_{\cdot}, S_{\cdot}) = WS_0 - CP(A_{\cdot}, S_{\cdot}).
            \end{equation*}
        \end{block}

    \end{frame}

    \begin{frame}{Almgren-Chriss}

            В случае AC модели, взяв в качестве процесса цены $S_l = p_b(l)$, получаем, что общие издержки торговой траектории выглядит следующим образом:
    
        \begin{equation*}
            \mathcal P = Wp_b(l) - \sum\limits_{l=1}^L p_b(l)A_l = \sum\limits_{l=1}^L ( g(A_l)w_l + h(A_l)A_l) - \sum\limits_{l=1}^L \sigma \zeta_l w_l .
        \end{equation*}

        Цель AC алгоритма состоит в решении следующей оптимизационной задачи 

        \begin{equation*}
            \E [ \mathcal P] + \lambda \var[\mathcal P] \to \min, \quad \lambda > 0.
        \end{equation*}

        Параметр $\lambda >0$ интерепретируется как степень неприятия риска.

    \end{frame}

    \begin{frame}{Almgren-Chriss}

        В случае когда $g(x) = \gamma x, \; h(x) = \eta x $, данная оптимизационная задача решается в явном виде и ее решение имеет следующий вид:

        \begin{align} 
            &A_l^{*} = \frac{2 \sinh(\kappa/2)}{\sinh(\kappa L)}\cosh(\kappa (L - l + 0.5) )W, \label{Al}\\
            &\kappa = \cosh^{-1}(0.5\tilde \kappa^2 + 1), \; \tilde \kappa = \frac{\lambda \sigma}{\eta - 0.5 \gamma}. \label{kappa}
        \end{align}

    \end{frame}


    \section{Online ML}

    \begin{frame}{Пару слов об OML и RL}

        Онлайн-машинное обучение, также известное как потоковое машинное обучение, позволяет осуществлять процесс обучения по мере поступления новых данных. В отличие от обычного ML, где алгоритм обучается на фиксированном наборе данных, алгоритмы OML обновляют свои модели автоматически, динамически включая новые наблюдения в процесс обучения. 
    
        Обучение с подкрепление это подобласть OML. Алгоритмы обучения с подкреплением учатся методом проб и ошибок при взаимодействии с окружающей средой, стремясь максимизировать сигнал вознаграждения.

    \end{frame}

    \begin{frame}{Пару слов об OML и RL}
        Мощным инструментом, который тесно связан с OML, и который используется для решения задач оптимального выполнения, являются марковские процессы принятия решений (MDP). MDP обеспечивают математическую основу для моделирования процесса принятия решений в стохастических средах, где результаты действий неопределенны.

        Формально, MDP определяется набором $(\mathcal S, \mathcal A, P, R)$, где
    
        \begin{itemize}
    
            \item $\mathcal S$ - предствляет собой набор состояний окружающей среды,
            
            \item $\mathcal A$ - представляет собой набор действий, которые могут быть предприняты,
    
            \item $P$ - функция переходов, которая определяет вероятность перехода из одного состояния в другое,
    
            \item $R$ - функция вознаграждения. которая количественно определяет желательность или полезность пребывания в определенном состоянии или совершения определенного действия.
        \end{itemize}
    
    \end{frame}

    \begin{frame}{GLOBE}

        Предположим, что мы хотим продать $W_1, W_2, \ldots, W_N$ лотов некоторого актива, за периоды времени $[0, T], [T, 2T], \ldots, [(N-1)T, NT]$ соотвественно. Такие пероды времени будем называть раундами.
        
        Как и раньше, каждый раунд делится еще на $L$ равных интервалов длины $\tau = T/L$, и аналогично тому, что было раньше, на каждом раунде $\rho$ определяются $(A_l^\rho)_{l \in \mathcal L}, \; (w_l^\rho)_{l \in \mathcal L \cup \{0\}}$.

    \end{frame}

    \begin{frame}{GLOBE}

        Введем следующие множества:
    
        \begin{itemize}
            \item $\mathcal M , \; |\mathcal M| < \infty$ - множество состояний рынка, которое будет использоваться для определения динамики бида, как марковского процесса с конечным числом состояний:
            \begin{equation*}
                \mathcal M \ni M_l^\rho = \frac {p_b(\rho, l) - p_b(\rho, 1)}{\sigma_\rho}.
            \end{equation*}
            
            \item $\mathcal I = \{0,1, 2, \ldots, W_{\max} \}$ - множество частных состояний или множество возможных значений $w_l^\rho$. Более того, предположим, что $W_j \in [W_{\min}, W_{\max}], \; 0<W_{\min} \leqslant W_{\max}, \; \rho \in \{1, 2, \ldots, N \}$. Частное состояние в слоте $l$ в раунде $\rho$ обозначим через $I_l^\rho \in \mathcal I$. Считаем, что $I_1^\rho = W_\rho$.
        \end{itemize}
    
    \end{frame}

    \begin{frame}{GLOBE}
        В качестве $\sigma_\rho$ берется
        \begin{align*}
            &\sigma_\rho = \sqrt{\frac {\sum\limits_{j=1}^{\rho -1}[Ret(j) - \mu_\rho ]^2} {\rho - 1} },\\
            &\mu_\rho =  \frac1{\rho -1}\sum_{j=1}^{\rho -1} Ret(j) , \quad Ret(j) =  \log(p_m(\rho, L)/p_m(\rho, 1) ).
        \end{align*}

         Также вводится множество действий на каждом слоте каждого раунда $\mathcal A_l^\rho = \{0,1,2,\ldots, A_l^\rho \}$, где $A_l^\rho = A_l^{*}$, которые находятся из уравнений \eqref{Al} - \eqref{kappa} в начале каждого раунда. 
    \end{frame}

    \begin{frame}{GLOBE}
        \begin{block}{Определение}
        Введем ошибку реализации:
            \begin{equation*}
                IS_\rho = 1 - \sum\limits_{l = 1}^{L} \frac{A_l^\rho p_b(\rho, l)}{Wp_m(\rho , 1)}.
            \end{equation*}
            Тогда средняя стоимость за раунд определяется как
            \begin{equation*}
                ACPR_\rho = \frac{1}{\rho} \sum\limits_{j=1}^\rho IS_j.
            \end{equation*}
        \end{block}

    \end{frame}

    \begin{frame}{GLOBE}

        Заметим, что используя введенные выше обозначения, ошибка реализации переписывается в виде
        
        \begin{equation*}
            IS_\rho = \sum\limits_{l=1}^L C_{X_\rho}(M_l^\rho, A_l^\rho), \quad C_{X_\rho}(M_l^\rho, A_l^\rho) = \frac{1}{W_\rho p_m(\rho, 1)} \left[ A_l^\rho \left(\frac{B_1^\rho}{2} - M_l^\rho \sigma_\rho \right) \right],
        \end{equation*}
        
        где $B_l^\rho = p_a(\rho, l) - p_b(\rho, l), \; X_\rho = (W_\rho, p_m(\rho, 1), \sigma_\rho, B_1^\rho )$ - определяются в начале раунда, и не меняются в течение него. Обозначим множество возможных значений $X_\rho$ через $\mathcal X$.
    
    \end{frame}

    \begin{frame}{GLOBE}
    
         Постановка задачи: на каждом раунде $\rho$ нахождение такой политики $\pi = (\pi_1, \pi_2, \ldots, \pi_L)$,
         \begin{equation*}
             \pi_l \colon \mathcal M \times \mathcal I \to \mathcal A_l^\rho ,
         \end{equation*}

        что она минимизирует
        \begin{equation*}
            \E \left[ C_{X_\rho}^\pi \right] = \E \left[ \sum\limits_{l=1}^L C_{X_\rho}(M_l^\rho, \pi_l (M_l^\rho, I_l^\rho) ) \right].
        \end{equation*}

        Данная задача решаетя в случае, если известна переходная функция $P$. Используя принцип оптимальности Беллмана, получаем, что решение данной оптимизационной задачи есть
        
         
    \end{frame}
    
    
    \begin{frame}{GLOBE}

        \begin{align*}
            &V_l(M, I) =\min\limits_{a \in \mathcal A_l^\rho}\left[ Q(M, I, a) \right] =\min\limits_{a \in \mathcal A_l^\rho} \left[C_X(M, a) + \sum\limits_{M^\prime \in \mathcal M} P(M, M^\prime) V_{l+1}(M^\prime, I - a ) \right], \\
            &\pi_l(M,I) = \arg\min\limits_{a \in \mathcal A_l^\rho}Q(M, I, a), \quad V_L(M,I) = C_X(M,I), \; \pi_L(M,I) = I,
        \end{align*}
        для любых $M \in \mathcal M, \; I \in \mathcal I, \; l \in \mathcal L \backslash \{L\}, \; X \in \mathcal X $.

        На практике данная задача решается методом динамического программирования. 

        Неизвестная функция $P(M, M^\prime)$ заменяется на эмпирический аналог $\hat P(M, M^\prime)$, который в начале каждого раунда обновляется следующим образом:

    \end{frame}

    \begin{frame}{GLOBE}
        \begin{align*}
            &N_\rho(M, M^\prime) = \sum\limits_{\rho^\prime = 1}^{\rho - 1} \sum\limits_{l = 1}^{L - 1} \mathbb I (M_{\rho^\prime}^l = M)\mathbb I (M_{\rho^\prime}^{l + 1} = M^\prime), \quad M, M^\prime \in \mathcal M, \\
            &N_\rho(M) = \sum\limits_{M^\prime \in \mathcal M} N_\rho(M, M^\prime), \quad M \in \mathcal M,\\
            &\hat P(M, M^\prime) = \frac{N_\rho(M, M^\prime) + \mathbb I (N_\rho(M) = 0 ) } {N_\rho (M) + |\mathcal M| \mathbb I (N_\rho(M) = 0 )} \quad M, M^\prime \in \mathcal M.
        \end{align*}

        $N_\rho(M, M^\prime), N_\rho(M)$ - обновляются в конце каждого раунда.

        Оказывается, что если $W_{\max}$ достаточно мал (например, объем, который необходимо продать в данном слоте всегда меньше объема LOB на уровне лучшего бида), то $\mathcal A_l^\rho$ можно заменить на $\mathcal A_l^{*,\rho} = \{0, A_l^\rho \}$.
        
    \end{frame}

    \begin{frame}{GLOBE+}
        Был рассмотрен еще один алгоритм. Все аналогично обычному алгоритму, который был описан выше, за исключением того, что $\mathcal A_l^\rho$ меняется на $\{0,1,\ldots 2A_l^\rho\}$.

        Само собой, теоретический результат с прошлого слайда в этом случае не применим.

    \end{frame}


    \section{Практическая реализация}

    % \begin{frame}{Построение стакана}
    %     Отметим, что все данные поступают в формате JSON, 
        
    % \end{frame}
    
    \begin{frame}{Метрики}
        Все алгоритмы будут сравниваться по "метрике" $ACPR$ и связанной с ней $RI$, которая определяется следующим образом

        \begin{block}{Определение}

            $RI$ показывает относительное снижение торговых
            издержек по сравнению с базовой моделью (TWAP, в нашем случае):
    
            \begin{equation*}
                RI_\rho(alg) = \frac{ACPR_\rho(baseline) - ACPR_{\rho}(alg)  }{|ACPR_\rho(baseline)|}.
            \end{equation*}
            
        \end{block}
        
    \end{frame}
    
    \begin{frame}{TWAP}
    
        Отметим, что объем и ценовые уровни целочисленные. Поэтому имеет смысл все $A^\rho_l$ моделировать целочисленными, т.е. $A^\rho_l = [W_\rho/L], \; l \in \mathcal L \backslash \{L\} , \; A^\rho_L = [W_\rho/L] + (W_\rho - L[W_\rho/L])$, где $[x]$ - целая часть $x$. 

        Никаких дополнительных параметров не вводится, кроме числа раундов, для сравнения данного алгоритма с GLOBE.
        
    \end{frame}

    \begin{frame}{AC}

        Аналогично предыдущему, $A^\rho_l = [A_l^{*, \rho}], \; l \in \mathcal L \backslash \{L\} , \; A^\rho_L = [A_L^{*, \rho}] + (W_\rho - \sum\limits_{l=1}^{l-1}A_l^\rho )$. В качестве дополнительных параметров на вход должны подаваться $\eta, \gamma$, а также число раундов, и число раундов для оценки $\sigma_\rho$ (напомним, что $\sigma_\rho$ можно определить лишь в начале второго раунда, и то в этом случае, она будет равна 0). 

    \end{frame}

    \begin{frame}{GLOBE и GLOBE+}
        В качестве $\mathcal A_l^{*, \rho}$ берется множество $\{0, [A_l^{*,\rho}]\}$ (для GLOBE+ множество действий: $\{0, 1, 2, \ldots, 2 [A_l^{*,\rho}] \} )$. Помимо параметров, которые передается в AC, передается также множество $\mathcal M$ которое оценивается по историческим данным. Во время оценки $\sigma_\rho$, множество $\mathcal M$ также будет обновляться. 
    
        Состояние рынка в слоте $l$ в раунде $\rho$ определяется как 
        \begin{equation*}
            M_l^\rho = \left[\frac {p_b(\rho, l) - p_b(\rho, 1)}{K\sigma_\rho} \right],
        \end{equation*}
        где $K \in \mathbb N$ передается в качестве дополнительного параметра, и нужен для того, чтобы уменьшить множество $\mathcal M$, мощность которого существенно влияет на сложность алгоритма.

    \end{frame}


    \section{Результаты}

    \begin{frame}{Параметры}
        Во всех тестах, если не оговорено иное, параметры брались одни и те же: $T = 50, W_\rho = 500, \; \gamma = 0, \; \eta = 0.002, \; \lambda = 0.02, \; N = 10, \; rounds\_for\_est = 15, \; K = 100\_000$. 

        При таком выборе $K$ число состояний в $\mathcal M$ порядка 10-20. Для алгоритма TWAP, первые 15 раундов пропускались. 
        
    \end{frame}
    
    \begin{frame}{Графики}
    
        \begin{figure}  
            \centering
            \includegraphics[width=0.83\linewidth]{USD_CNH_T+1 2022-10-04 T = 50 W = 500}
            \caption{USD\_CNH\_T+1 2022-10-04. }
        \end{figure}

    \end{frame}

    \begin{frame}{Графики}
    
        \begin{figure}  
            \centering
            \includegraphics[width=0.83\linewidth]{USD_CNH_T+1 2022-11-15 T = 50 W = 500}
            \caption{USD\_CNH\_T+1 2022-11-15. }
        \end{figure}

    \end{frame}

    \begin{frame}{Графики}
    
        \begin{figure}  
            \centering
            \includegraphics[width=0.83\linewidth]{USD_RUB_T+1 2022-10-20 T = 50 W = 500}
            \caption{USD\_RUB\_T+1 2022-11-15. }
        \end{figure}

    \end{frame}

    \begin{frame}{Графики}
    
        \begin{figure}  
            \centering
            \includegraphics[width=0.83\linewidth]{USD_RUB_T+1 2022-10-28 T = 50 W = 500}
            \caption{USD\_RUB\_T+1 2022-10-28. }
        \end{figure}

    \end{frame}


    \begin{frame}{Графики}
    
        А теперь немного о том, что происходит, если брать объем сильно больше объема на уровне лучшего бида:  $W = 500\_000$ ( сбивает 4 ценовых уровня).

    \end{frame}


    \begin{frame}{Графики}

        \begin{figure}  
            \centering
            \includegraphics[width=0.83\linewidth]{USD_RUB_T+1 2022-10-04 T = 50 W = 500, x10000}
            \caption{USD\_RUB\_T+1 2022-10-04. }
        \end{figure}
        
    \end{frame}

    \begin{frame}{Графики}

        \begin{figure}  
            \centering
            \includegraphics[width=0.83\linewidth]{USD_CNH_T+1 2022-10-04 T = 50 W = 500 x1000}
            \caption{USD\_CNH\_T+1 2022-10-04. }
        \end{figure}
        
    \end{frame}


    \begin{frame}{Пару слов о графиках}
        Было замечено, что алгоритмы довольно чуствительны к выбору параметров, однако, как их оценивать, авторам пока не очевидно...

        Также стоит отметить, что более сложные алгоритмы чаще оказывались лучше при малых объемах. 

        При больших объемах ситуация неоднозначная. Возможно, параметры моделей зависят от объема. 
        
    \end{frame}

    \section{Заключение}

    \begin{frame}{Подведение итогов}

        Была проведена работа по изучению абсолютно новой (для авторов) темы. 
        Остальная часть работы носила чисто прикладной характер.
        Реализовано и протестировано на реальных данных 3.5 алгоритма (TWAP,AC, GLOBE, GLOBE+). Эмпирически получено, что более сложные алгоритмы, все же, в среднем, дают небольшой выигрышь. 
        
        
        
    \end{frame}

    \begin{frame}{Что дальше?}
        \begin{itemize}
            \item Придумать и реализовать алгоритмы калибровки параметров для AC и GLOBE.
            \item Реализвоать и протестировать другие RL алгоритмы (например, основаные на Q-обучении).
            \item Возможно, попробовать наиболее удачные алгоритмы на реальном рынке. 
            \item Придумать (или хотя бы попытаться), как можно улучшить данные алгоритмы. Возможно придумать собственный.
        \end{itemize}
    \end{frame}

    \begin{frame}{Послесловие}
        Немного о том, как шла работа в последние 2 дня

        \begin{figure}  
            \centering
            \includegraphics[width=0.83\linewidth]{workflow.jpeg}
            % \caption{USD\_CNH\_T+1 2022-10-04. }
        \end{figure}
    \end{frame}

    

    
\end{document}